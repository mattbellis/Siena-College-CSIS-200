%%%%%%%%%%%%%%%%%%%%%%%%%%%%%%%%%%%%%%%%%%%%%%%%%%%%%%%%%%%%%%%%%%%%%%%%%%%%%%%%
\documentclass[11pt]{article}

%%%%%%%%%%%%%%%%%%%%%%%%%%%%%%%%%%%%%%%%%%%%%%%%%%%%%%%%%%%%%%%%%%%%%%%%%%%%%%%%
% Note that comments begin with a "%" and are not turned into text in the .pdf
% document.
%%%%%%%%%%%%%%%%%%%%%%%%%%%%%%%%%%%%%%%%%%%%%%%%%%%%%%%%%%%%%%%%%%%%%%%%%%%%%%%%

%%%%%%%%%%%%%%%%%%%%%%%%%%%%%%%%%%%%%%%%%%%%%%%%%%%%%%%%%%%%%%%%%%%%%%%%%%%%%%%%
% Include some extra packages and change the defaults.
%%%%%%%%%%%%%%%%%%%%%%%%%%%%%%%%%%%%%%%%%%%%%%%%%%%%%%%%%%%%%%%%%%%%%%%%%%%%%%%%
\usepackage[]{graphicx}

\hsize=6.5in
\hoffset=-0.75in
\setlength{\textwidth}{6.5in}
\setlength{\textheight}{9in}
\setlength{\voffset}{10pt}
\setlength{\topmargin}{0pt}
\setlength{\headheight}{0pt}
\setlength{\headsep}{0pt}
%%%%%%%%%%%%%%%%%%%%%%%%%%%%%%%%%%%%%%%%%%%%%%%%%%%%%%%%%%%%%%%%%%%%%%%%%%%%%%%%

\title{Of Dogs and Lizards: a Parable of Privilege\footnote{{\tt https://sindeloke.wordpress.com/2010/01/13/37/}}}

%%%%%%%%%%%%%%%%%%%%%%%%%%%%%%%%%%%%%%%%%%%%%%%%%%%%%%%%%%%%%%%%%%%%%%%%%%%%%%%%
\begin{document}
\maketitle

Today I'm feeling 101-y, I guess, so let's talk about privilege.

It's a weird word, isn't it? A common one in my circles, it's one of the most basic, everyday concepts in social activism, we have lots of unhelpful snarky little phrases we like to use like ``check your privilege" and a lot of our dialog conventions are built around a mutual agreement (or at least a mutual attempt at agreement) on who has privilege when and how to compensate for that. But nonetheless fairly weird, opaque even if you've never used it before or aren't part of those circles. It's also, the way we use it, very much a cultural marker - like ``Tolkienesque" or ``Hall-of-famer" or ``heteronormative," you can feel fairly assured that a large number of people will immediately stop listening and stop taking you seriously the moment you use it.

The fact that people are stupid isn't news, however. And actually that's kind of why the concept of privilege is important - because privilege isn't {\it about} being stupid. It's not a bad thing, or a good thing, or something with a moral or value judgement of any kind attached to it. Having privilege isn't something you can usually change, but that's okay, because it's not something you should be ashamed of, or feel bad about. Being told you have privilege, or that you're privileged, isn't an insult. It's a reminder! The key to privilege isn't worrying about having it, or trying to deny it, or apologize for it, or get rid of it. It's just paying attention to it, and knowing what it means for you and the people around you. Having privilege is like having big feet. No one hates you for having big feet! They just want you to remember to be careful where you walk.

At this point maybe I should actually start talking about what privilege {\it is} , huh?

Well, we're right here online, so let's start with the Google definition. As per standard for googledefs, it's hardly comprehensive, but entirely adequate for our purposes here, particularly the second entry:

\begin{quote}If you talk about {\bf privilege}, you are talking about the power and advantage that only a small group of people have, usually because of their wealth or their high social class.
\end{quote}
This is the basic heart of the idea. Privilege is an edge... a set of opportunities, benefits and advantages that some people get and others don't. For example, if it's raining in the morning, and you get up, get dressed, climb into the nice warm car in your garage, drive to the closed parking lot at work, and walk into the adjacent building, you don't get wet. If you go outside and wait at the bus stop, then walk between busses for your transfer, then walk from the bus stop to work, you do get wet. Not getting wet, then, is a privilege afforded you by car and garage ownership. So far, so straightforward, right?

Some examples of social privilege work exactly the same way, and they're the easy ones to understand. For instance, a young black male driver is much, much more likely to get pulled over by the cops in America than an old white woman. Getting pulled over less, then - being given the benefit of the doubt by an authority figure - is in this case, a privilege of being white. (I'm not getting into the gender factor here, intersectionality is a whole different post.)

Okay, again, so far so straightforward. And thus far, there's not much to be done about it, right? You're not going to, as a white person, make a point of getting pulled over more often, and nobody's asking you to. (Well, I'm not, at least.) So if someone says ``check your privilege," if I tell you to watch where you're putting your feet, what the hell does that mean?

Well. This is where things get a bit tricky to understand. Because {\it most}  examples of social privilege aren't that straightforward. Let's take, for example, a basic bit of male privilege:

{\it A man has the privilege of walking past a group of strange women without worrying about being catcalled, or leered at, or having sexual suggestions tossed at him.} 

A pretty common male response to this point is ``that's a privilege? I would {\it love}  if a group of women did that to me."

And that response, right there, is a {\it perfect shining example}  of male privilege.

To explain how and why, I am going to throw a lengthy metaphor at you. In fact, it may even qualify as parable. Bear with me, because if it makes everything crystal clear, it will be worth the time.

Imagine, if you will, a small house, built someplace cool-ish but not cold, perhaps somewhere in Ohio, and inhabited by a dog and a lizard. The dog is a big dog, something shaggy and nordic, like a Husky or Lapphund - a sled dog, built for the snow. The lizard is small, a little gecko best adapted to living in a muggy rainforest somewhere. Neither have ever lived anywhere else, nor met any other creature; for the purposes of this exercise, this small house is the entirety of their universe.

The dog, much as you might expect, turns on the air conditioning. Really cranks it up, all the time - this dog was bred for hunting moose on the tundra, even the winter here in Ohio is a little warm for his taste. If he can get the house to fifty (that's ten C, for all you weirdo metric users out there), he's almost happy.

The gecko can't do much to control the temperature - she's got tiny little fingers, she can't really work the thermostat or turn the dials on the A/C. Sometimes, when there's an incandescent light nearby, she can curl up near it and pick up some heat that way, but for the most part, most of the time, she just has to live with what the dog chooses. This is, of course, much too cold for her - she's a gecko. Not only does she have no fur, she's cold-blooded! The temperature makes her sluggish and sick, and it permeates her entire universe. Maybe here and there she can find small spaces of warmth, but if she ever wants to actually do anything, to eat or watch TV or talk to the dog, she has to move through the cold house.

Now, remember, she's never known anything else. This is just how the world is - cold and painful and unhealthy for her, even dangerous, and she copes as she knows how. But maybe some small part of her thinks, ``hey, it shouldn't be like this," some tiny growing seed of rebellion that says who she is right next to a lamp is who she should be all the time. And she and the dog are partners, in a sense, right? They live in this house together, they affect each other, all they've got is each other. So one day, she sees the dog messing with the A/C again, and she says, ``hey. Dog. Listen, it makes me really cold when you do that."

The dog kind of looks at her, and shrugs, and keeps turning the dial.

This is not because the dog is a jerk.

This is because the dog has {\it no fucking clue what the lizard even just said} .

Consider: he's a nordic dog in a temperate climate. The word ``cold" is {\it completely meaningless}  to him. He's never been cold in his entire life. He lives in an environment that is perfectly suited to him, completely aligned with his comfort level, a world he grew up with the tools to survive and control, built right in to the way he was born.

So the lizard tries to explain it to him. She says, ``well, hey, how would you like it if I turned the temperature down on you?"

The dog goes, ``uh... sounds good to me."

What she really means, of course, is ``how would you like it if I made you cold." But she can't make him cold. She doesn't have the tools, or the power, their shared world is not built in a way that allows it - she simply is not physically capable of doing the same harm to him that he's doing to her. She could make him feel pain, probably, I'm sure she could stab him with a toothpick or put something nasty in his food or something, but this specific form of pain, he will never, ever understand - it's not something that {\it can}  be inflicted on him, given the nature of the world they live in and the way it's slanted in his favor in this instance. So he doesn't get what she's saying to him, and keeps hurting her.

Most privilege is like this.

A straight cisgendered male American, because of who he is and the culture he lives in, does not and cannot feel the stress, creepiness, and outright threat behind a catcall the way a woman can. His upbringing has given him fur and paws big enough to turn the dials and plopped him down in temperate Ohio. When she says ``you don't have to put up with being leered at," what she means is, ``{\it you don't ever have to be wary of sexual interest} ." That's male privilege. Not so much that something doesn't happen to men, but that it will never carry the same weight, even if it does.

So what does this mean? And what are we asking you to do, when we say ``check your privilege" or ``your privilege is showing"?

Well, quite simply, we want you to understand when you have fur. And, by extension, when that means you should {\it listen} . See, the dog's not an asshole just for turning down the temperature. As far as he knows, that's fine, right? He genuinely cannot feel the pain it causes, he doesn't even know about it. No one thinks he's a bad person for totally accidentally doing harm.

Here's where he becomes an asshole: the minute the gecko says, ``look, you're hurting me," and he says, ``what? No, I'm not. This 'cold' stuff doesn't even exist, I should know, I've never felt it. You're imagining it. It's not there. It's fine because of fur, because of paws, because look, you can curl up around this lamp, because sometimes my water dish is too tepid and I just shut up and cope, obviously temperature isn't this big deal you make it, and you've never had to deal with mange anyway, my life is just as hard."

And then the dog just ignores it. Because he can. That's the privilege that comes with having fur, with being a dog in Ohio. He doesn't have to think about it. He doesn't have to live daily with the cold. He has {\it no idea what he's talking about} , and he will never, ever be forced to learn. He can keep making the lizard miserable until the day they both die, and he will never suffer for it beyond the mild annoyance of her complaining. And she, meanwhile, gets to try not to freeze to death.

So, quite simply: don't be that dog. If you're straight and a queer person says ``do not title your book 'Beautiful Cocksucker,' that's stupid and offensive," {\it listen and believe him} . If you're white and a black person says ``really, now, we're all getting a little tired of that What These People Need Is A Honky trope, please write a better movie," {\it listen and believe her.}  If you're male and a woman says ``this maquette is a perfect example of why women don't read comics," {\it listen and believe her.}  Maybe you don't see anything wrong with it, maybe you think it's oh-so-perfect to your artistic vision, maybe it seems like an oversensitive big deal over nothing to you. WELL OF COURSE IT DOES, YOU HAVE FUR. Nevertheless, just because you personally can't feel that hurt, doesn't mean it's not real. All it means is you have privilege.

That's not a bad thing. You can't help being born with fur. Every single one of us has some kind of privilege over somebody. What matters is whether we're aware of it, and what we choose to do with it, and that we not use it to dismiss the valid and real concerns of the people who don't share our particular brand.

\end{document}
%%%%%%%%%%%%%%%%%%%%%%%%%%%%%%%%%%%%%%%%%%%%%%%%%%%%%%%%%%%%%%%%%%%%%%%%%%%%%%%%
